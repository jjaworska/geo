\documentclass[12pt,a4paper]{article}
\usepackage{geometry}
\newgeometry{tmargin=1.5cm, bmargin=1.5cm, lmargin=2cm, rmargin=2cm}
\usepackage[fleqn]{amsmath}
\usepackage{amssymb}
\usepackage{mathtools}
\usepackage{mathabx}
\let\degree\relax
\usepackage{textcomp}
\usepackage{gensymb}
\usepackage[T1]{fontenc}
\usepackage[polish]{babel}
\usepackage[utf8]{inputenc}
\usepackage{lmodern}
\usepackage{color}
\usepackage{fancyhdr}
\usepackage[lastpage,user]{zref}
\pagestyle{fancy}
\cfoot{\thepage/\zpageref{LastPage}}
\allowdisplaybreaks
\usepackage[shortlabels]{enumitem}

\begin{document}
\begin{center}
    \large{\textbf{Cudowne rzeczy z geometrii $\heartsuit$}}
\end{center}

%\renewcommand{\labelitemi}{$\heartsuit$}

\begin{small}
\begin{itemize}
    \item Okręgi, dużo okręgów...
    
    Warunki na wpisywalność czworokąta w okrąg:
    \begin{itemize}
        \item kątowe
        \item potęgowe
    \end{itemize}
    
    Przypadek zdegenerowany - okrąg styczny (dostajemy tu tw. o kącie między styczną a cięciwą)
    
    \item Twierdzenia Talesa (innymi słowy myślenie o rysunku w kontekście jednokładności)
    
    \item Przekszałcenia płaszczyzny! (symetria, obrót, translacja, izometria, inwersja*)
    
    \item Podobieństwo spiralne (brzmi strasznie, ale jest proste)
    
    \item podobieństwo zwykłe, przystawanie trójkątów
    
    \item osie potęgowe (fakt, że jest ona prostopadła do prostej łączące środki okręgów) (Tw. Monge'a*)
    \newline
    
    Co się dzieje w trójkącie?
    
    \item Odbicie ortocentrum w dowolnym boku leży na okręgu opisanym
    
    \item ortocentrum i środek okręgu opisanego są sprzężone izogonalnie
    
    \item wzory na długości odcinków od wierzchołka trójkąta do punktu styczności z okręgiem wpisanym/dopisanym
    
    \item twierdzenie o trójliściu (czwórliściu*)
    
    \item twierdzeniu o dwusiecznej (również zewnętrznej*)
    
    \item świetny lemat o punkcie styczności okręgu wpisanego*
    
\end{itemize}
\end{small}
\begin{small}
\textbf{Jak podejść do zadania?}
\begin{itemize}
    \item Starać się zawsze przepisać warunek  tezę zadania na bardziej ``operatywny'' - jak w zadaniu są podane iloczyny odcinków, to albo chcemy z nich stworzyć jakąś potęgę punktu albo przpisać na jakieś podobieństwo
    
    \item Szukać ``magicznych punktów'', czasem się narzucają (np przez stworzenie jakiegoś trójkąta przystającego w innym miejscu), a czasem warto pomyśleć, skąd możnaby taki punkt wyczarować (np przenieść gdzieś jakiś odcinek i poszukać własności)
    
    \item Dokładne rysunki! - fajne claimy często po prostu się ``czyta z rysunku''. I odwrotnie - zamiast starać się udowodnić jakieś stwierdzenie, można od razu stwierdzić, że nie ma szans
    
    \item tip: rysunek warto zacząć od narysowania okręgu, nawet jeśli pojawia się on w treści później. (Okrąg opisany na trójkącie warto mieć nawet jeśli nie ma go w treści)
\end{itemize}
\end{small}

\end{document}

